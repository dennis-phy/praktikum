\section{Diskussion}
\label{sec:Diskussion}

\begin{table}[h]
	\centering
	\caption{Vergleich der experimentell bestimmten Werte mit der Theorie}
	\label{tab:diskussion}
	\begin{tabular}{ S | S S S }
		\toprule
		{  } & { Theoriewert für $k$} & {Exp. ermittelter Wert} & {rel. Abweichung}\\
		\midrule
            \text{Rechteckschwingung} & -1 & \num{-0.923 \pm 0.019} & 7.7 \% \\ 
            \text{Dreieckschwingung} & -2 & \num{-1.976 \pm 0.018} & 1.2 \% \\
            \text{Sägezahnschwingung} & -1 & \num{-0.867 \pm 0.031} & 13.3 \% \\
	\end{tabular}
\end{table}
In Tabelle \ref{tab:diskussion} werden die experimentell ermittelten Werte für die Exponenten mit den theoretischen verglichen.Die Theoriewerte wurden dabei aus den in der Vorbereitung ermittelten Gleichungen herausgelesen. Die Abweichungen vom Theoriewert können darauf zurückgeführt werden, dass es immer eine Grundspannung in der Messapparatur gab. Beim Bestimmen der Höhen der Peaks gab es stets Schwankungen, sodass die Werte nur abgeschätzt werden konnten. Die Messung für die Dreieckschwingung lieferte die besten Ergebnisse. Ein möglicher Grund dafür ist die Abhängigkeit von $\frac{1}{n^2}$. Da die Werte schnell abfallen, fallen die Werte für höhere $n$, welche schwerer abzulesen sind, weniger stark ins Gewicht.
Trotz der Abweichungen lassen sich allerdings einige Eigenschaften des Fourier-Theorems, wie etwa die nicht vorhandenen Peaks bei geraden $n$ bei der Rechteck- und Dreieckschwingung, klar nachweisen. Ebenso stehen die aufgenommenen Bilder zur Fouriersynthese in guter Übereinstimmung mit der theoretischen Vorhersage.
