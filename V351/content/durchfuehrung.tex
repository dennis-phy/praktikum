\section{Durchführung}
\label{sec:Durchführung}
\subsection{Versuchsaufgabe}
Das Ziel des ersten Teils der Versuchsaufgabe ist das Zusammensetzen einer
periodischen Schwingung aus ihren Fourier-Komponenten sodas die gewünschte
Schwingung(Rechteck, Sägezahn, Dreieck) entsteht. Im zweiten Teil des Versuches
soll dies dann umgekehrt werden und mittels einer Fourier-Analyse eine gegebene
periodische Funktion in ihre Fourier-Koeffizienten zerlegt werden.
\subsection{Aufbau und Durchführung der Fourier-Synthese}
Für die Fourier-Synthese wurde ein Oberwellengenerator verwendet, der bis zu 10
Oberwellen erzeugen kann. Diese können alle bezüglich ihrer Amplitude und Phase
separat eingestellt werden. Als erstes muss nun die Phasengleichheit der
Oberschwingung mit den einzelnen Oberwellen eingestellt werden. Dazu werden
jeweils die Oberschwingung sowie eine der Oberwellen an zwei Eingänge des
Oszilloskops angeschlossen und im XY-Modus dargestellt. Im XY-Modus führ das
Signal an Eingang 1 zu einer Auslenkung in X-Richtung und das Signal an Eingang
2 zu einer Auslenkung in Y-Richtung. in dieser Darstellung bilden sich bei
Periodischen Funktionen sogenannte Lissajous-Figuren\cite{V351}.
Lissajous-Figuren sind bei geraden Oberwellen bei Phasengleichheit Linien und
bei einer Verschiebung um 90° symmetrische "Schleifen". Bei ungeraden
Oberwellen sind bei Phasengleichheit symmetrisch und bei einer Verschiebung um
90° Linien. Wenn die Phasen richtig eingestellt sind müssen die Spannungen der
Oberwellen mittels des Multimeters entsprechend der Fourier-Koeffizienten der
gewünschten Funktion im Verhältnis zur Oberschwingung eingestellt werden.\\ Nun
muss der Summen-Ausgang des Oberwellengenerator an das Oszilloskop
angeschlossen werden und die Oberwellen nacheinander an geschaltet werden.
Falls es beim Anschalten der Oberwelle zu einer Verschlechterung des Signals
kommt muss die Oberwelle noch um 180° verschoben werden.
\subsection{Aufbau und Durchführung der Fourier-Analyse}
Der Aufbau der Fourier-Analyse setzt sich aus einem Funktionsgenerator und
einem damit verbundenen digitalem Oszilloskop zusammen. MIttels des
Funktionsgenerator lassen sich viele Grundsignale, wie Rechteck, Dreieck oder
Sägezahn, mit einstellbarer Frequenz und Amplitude ausgeben. Der
Funktionsgenerator ist mit einem BNC-Kabel am Oszilloskop angeschlossen, sodass
das Signal am Oszilloskop analysiert werden kann. Das digitale Oszilloskop ist
in der Lage den Frequenzbereich über eine FFT darzustellen.\\ \noindent Wenn
alles verbunden und eingeschaltet ist lassen sich im FFT-Menü des Oszilloskops
mittels des Cursors die Amplituden der einzelnen berechneten
Fourier-Koeffizienten in dB ablesen.