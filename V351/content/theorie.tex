\section{Theorie}
\label{sec:Theorie}

\subsection{Fourier-Theorie}
Das Fouriersche Theorem besagt, dass sich jedes periodische Signal $f(t)$ als
eine Reihe von Sinus- und Cosinusfunktionen der Form
\begin{equation}
  \label{eqn:Fourier-Reihe}
  f(t)=\frac{1}{2}a_0+\sum_{n=1}^{\infty}\left(a_n \cos(n\frac{2\symup{\pi}}{T}t)+b_n \sin(n\frac{2\symup{\pi}}{T}t)\right)
\end{equation}
beschreiben lässt. Wobei T die Periodendauer des Signals ist.
Mit der Periodendauer T lässt sich zur Grundfrequenz $\frac{1}{\symup{T}}$ eine Kreisfrequenz $\omega=\frac{2\symup{\pi}}{T}$
definieren. Die Koeffizienten $a_n$ und $b_n$ beschreiben die Anteile aller
Oberwellen, Schwingungen mit einem ganzzahligen Vielfachen der Grundfrequenz,
die im Signal enthalten sind. Analytisch lassen Sie sich mit den Formeln
\begin{align}
  a_n & =\frac{2}{T}\int_0^Tf(t)\cos{(n\frac{2\symup{\pi}}{T}t)}\symup{d}t   &
  b_n & =\frac{2}{T}\int_0^{T}f(t)\sin{(n\frac{2\symup{\pi}}{T}t)}\symup{d}t
  \label{eqn:Koeffizienten}
\end{align}
berechnen. Ein Anteil für n=0 entspricht dabei einem sogenannten Offset.
Das Signal ist im Mittel nicht 0, sondern schwingt um einen von 0
versetzten Wert.\\

\noindent Diese Reihendarstellung des Signals konvergiert gleichmäßig
gegen das zu beschreibende Signal $f(t)$, wenn $f(t)$ stetig ist. An Stellen der Unstetigkeit
tritt das Gibbsche Phänomen auf. Demnach kommt es im Bereich der Unsteigkeit zu einer
Überschwingung, die keinem experimentellen Fehlern zugrunde liegt.
\cite{V351}
\subsection{Fourier-Analyse}
Bei der Fourier-Analyse wird ein zeitlich periodisches Signal auf seine
Frequenzanteile geprüft. Dazu werden analoge und digitale Verfahren entworfen.
Ein analoges Verfahren basiert auf den Eigenschaften eines elektrischen
Schwingkreises. Bei einer speziellen Frequenz, der Resonanzfrequenz, besitzt
dieser einen merklichen Überschwinger. Wird ein Signal an den Eingang einer
solchen Schaltung gelegt und die Grundfrequenz auf Resonanzfrequenz gestellt,
können die Amplitudenverhältnisse der einzelnen Oberwellen zur Grundfrequenz
gemessen werden. Da bei neu eingestellter Frequenz $\nu/n$, die n-te Oberwelle
Resonanz am Ausgang hervorruft.\\

\noindent Bei digitalen Verfahren wird das Signal quantisiert. Mit der sogenannten
Abtastfrequenz, die mindestens doppelt so groß wie die höchste zu messende Oberwellenfrequenz sein soll,
werden Messwerte genommen und dann mit einer Fast-Fourier-Transformation (FFT) in den
Frequenzbereich übertragen, wo die Frequenzanteile dann abgelesen werden können.\\

\noindent Die Amplitudenverhältnisse werden teilweise in Dezibel angegeben. Um diese Werte in Volt umzurechnen lässt sich die im Praktikum genannte Formel
\begin{equation}
  V=10^{\frac{dB}{20}}
\end{equation}
nutzen.
\subsection{Fourier-Synthese}
In der Fouriersynthese werden periodische Signale aus bereits bestehenden
Oberwellen zusammengesetzt. Das heißt die Amplituden, also die
Fourierkoeffizienten, lassen sich so einstellen, dass ein gewünschtes
Ausgangssignal entsteht (z.B. Rechtecksignal). \cite{V351}